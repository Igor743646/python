% размер шрифта: 14pt
% и указываем размер страницы - у нас это A4 (a4paper)
% подключаем макеты оформления extreport или extarticle, 
% чтобы кегль выше 12pt отображался адекватно
\documentclass[14pt, a4paper]{extarticle}

%поля: 
%слева: 2.5
%справа: 1.5
%сверху: 2.5
%снизу: 2.5
\usepackage[left=2.5cm, right=1.5cm, top=2.5cm, bottom=2.5cm]{geometry}

% установка междустрочного интервала
% закомментировано, т.к. нас устраивает пока настройки по умолчанию
% \linespread{1.3}

% кодировка файла
% не должно объявляться более одного раза
% кодировку подключаем в первую очередь
% иначе могут быть весьма неприятные непонятные ошибки
\usepackage[utf8x]{inputenc}
% шрифты
\usepackage[T2A]{fontenc}
% языки
\usepackage{ upgreek }

\usepackage[english,russian]{babel}


% пакеты
\usepackage{amsmath} % align
\usepackage{amssymb}
\usepackage[usenames]{color}

% мета-данные. Т.е. данные, описывающие данные.
% в текущем случае не выводятся никуда (не будут видны в pdf)
\title{Курсовой проект по дисциплине Математический практикум}
\author{Выполнил: Автор, группа}

% подключение пакетов для ссылок и для SageTeX
\usepackage{hyperref}
\usepackage{sagetex}

% размер отступа для первой строки абзаца
\setlength{\sagetexindent}{10ex}

% определяем короткий псевдоним для объявления секции с автоматической нумерацией
\renewcommand{\thesection}{\number\numexpr\value{section}-1\relax}
% для subsection, например, задаём формат записи заголовка "номер секции.номер подсекции-1"
\renewcommand{\thesubsection}{\thesection.\number\numexpr\value{subsection}-1\relax}
\renewcommand{\thesubsubsection}{\thesubsection.\number\numexpr\value{subsubsection}-1\relax}

% установка начальных значений счетчиков секций и пр.
\setcounter{secnumdepth}{1}
% \setcounter{chapter}{1}  % этого счетчика нет в шаблоне 'article'
\setcounter{section}{1}


% открываем тег документа
\begin{document}

% все содержимое должно быть внутри
% Т.е.: \begin{document} ... здесь весь ваш курсовой проект, его содержимое и оформление \end{document}

% Чтобы было удобнее - разобьём содержимое на отдельные файлы
% для этого есть \input
% Кроме \input, для вставки в документ текстовых файлов используется команда \include.
% Она эквивалентна конструкции:
%\clearpage
%\input{file}
%\clearpage
% т. е. помещенный в файл фрагмент документа будет начинаться с новой страницы. Поэтому с помощью \include удобно вставлять файлы с отдельными главами документа. При этом расширение .tex нужно опустить. Кроме того, \include, в отличие от \input, нельзя использовать рекурсивно, т. е. нельзя помещать include'ы в уже вложенные файлы.

%Обложка
% внутри файла все необходимые настройки. Вроде отключения нумерации страниц для титульного листа
\include{"TemplateParts/TitlePage"}

% Остальное содержимое
\setcounter{page}{2} % начать нумерацию с номера 2
%\tableofcontents  % вставить автоматически генерируемое оглавление для файла
% обратите внимание, что нужно запускать сборку tex-файла не один раз, а два, чтобы увидеть изменения в оглавлении
% потому какв первый проход LaTeX парсит весь файл и получает информацию, касательно где и что
% но файл парсится (обрабатывается) последовательно, потому сразу заполнить содержимое \ оглавление невозможно
% во вторую сборку обновляется оглавление

% вставляем пример внедрения графика Sage в документ LaTeX
%\include{"TemplateParts/SagePlot"}
\begin{sagesilent}
for i in range(10):
	show(LatexExpr(i))
\end{sagesilent}
% вставляем пример с матрицами
% \include{"TemplateParts/Task2"}
% \include{"TemplateParts/Task8"}
\include{"TemplateParts/Main"}

% вставка списка источников
\include{"TemplateParts/Sources"}

% конец документа - закрываем тег
\end{document}
