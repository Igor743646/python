В прямоугольной системе координат $Oxyz$ поверхность второго порядка задана уравнением:

\begin{center}
$-2y^2 + 4yz - 3z^2 + 4y + 4z - 12 = 0$
\end{center}

\begin{sagesilent}
var("x y z")
f = -2*y^2 + 4*y*z - 3*z^2 + 4*y + 4*z - 12
\end{sagesilent}

\begin{center}
  \sageplot[width = 13cm][png]{implicit_plot3d(f, (x, 0, 8), (y, 0, 8), (z, 0, 8))}
\end{center}

Применяем алгоритм составления канонического уравнения поверхности второго порядка к каноническому виду.
Определяем коэффициенты: 
$a_{11} = 0, a_{12} = 0, a_{13} = 0, 
a_{22} = -2, a_{23} = 2, a_{33} = -3, 
a_1 = 0, a_2 = 2, a_3 = 2, 
a_0 = -12$.\\


1. Вычисляем инварианты:
\begin{center}
	$\uptau_1 = 0 - 2 - 3 = -5$,
	$\uptau_2 = \begin{vmatrix}
		  0 & 0\\
		  0 & -2
		\end{vmatrix} + \begin{vmatrix}
		  0 & 0\\
		  0 & -3
		\end{vmatrix} + \begin{vmatrix}
		  -2 & 2\\
		  2 & -3
		\end{vmatrix} = 2$. \\


	$\updelta = \begin{vmatrix}
		  0 & 0 & 0\\
		  0 & -2 & 2\\
		  0 & 2 & -3
		\end{vmatrix} = 0$,
	$\Updelta = \begin{vmatrix}
		  0 & 0 & 0 & 0\\
		  0 & -2 & 2 & 2\\
		  0 & 2 & -3 & 2\\
		  0 & 2 & 2 & -12\\
		\end{vmatrix} = 0$. \\


	$\upkappa_2 = \begin{vmatrix}
		  0 & 0 & 0\\
		  0 & -2 & 2\\
		  0 & 2 & -12
		\end{vmatrix} + \begin{vmatrix}
		  0 & 0 & 0\\
		  0 & -3 & 2\\
		  0 & 2 & -12
		\end{vmatrix} + \begin{vmatrix}
		  -2 & 2 & 2\\
		  2 & -3 & 2\\
		  2 & 2 & -12
		\end{vmatrix} = 12$. \\
\end{center}

2. Так как $\updelta = \Updelta = 0, \uptau_2 > 0,  \uptau_1\upkappa_2 < 0$, то
определяем, что данное уравнение задает эллиптический цилиндр

3. Составляем матрицу квадратичной формы и столбец коэффициентов линейной формы:

\begin{center}
	$ A = 
	\begin{pmatrix}
	  0 & 0 & 0\\
	  0 & -2 & 2\\
	  0 & 2 & -3
	\end{pmatrix}
	, a = 
	\begin{pmatrix}
	  0 \\
	  2 \\
	  2 
	\end{pmatrix}
	, a_0 = -12$
\end{center}

4. Составляем характеристическое уравнение:

\begin{center}
	\begin{equation*}
	det(A) = 
	\begin{vmatrix}
	  0-\lambda & 0 & 0\\
	  0 & -2-\lambda & 2\\
	  0 & 2 & -3-\lambda
	\end{vmatrix} = 0
	\end{equation*}
\end{center}

Получается: $\lambda(\lambda^{2}+5\lambda+2) = 0$

Корни:
\begin{center}
	$\lambda_{1} = \frac{\sqrt{17} - 5}{2}$\\
	$\lambda_{2} = -\frac{\sqrt{17} + 5}{2}$\\
	$\lambda_{3} = 0$
\end{center}

5. Вычислим коэффициенты канонического уравнения эллиптического цилиндра:

\begin{center}
	$a^{2} = -\frac{12}{2\frac{\sqrt{17} - 5}{2}}$, 
	$b^{2} = -\frac{12}{-2\frac{\sqrt{17} + 5}{2}}$
\end{center}

Таким образом, каноническое уравнение заданной поверхности имеет вид:

\begin{center}
	$\frac{(x')^{2}}{ \sqrt{\frac{3(5+\sqrt{17})}{2}}^{2} } +
	 \frac{(y')^{2}}{ \sqrt{\frac{3(5-\sqrt{17})}{2}}^{2} } = 1$
\end{center}

6. Найдем собственные векторы:

Для $\lambda_1 = \frac{\sqrt{17} - 5}{2}$: $l_1 = \begin{pmatrix}
	  0 \\
	  1+\sqrt{17} \\
	  4 
	\end{pmatrix}$

Для $\lambda_2 = -\frac{\sqrt{17} + 5}{2}$: $l_1 = \begin{pmatrix}
	  0 \\
	  1-\sqrt{17} \\
	  4 
	\end{pmatrix}$

Для $\lambda_3 = 0$: $l_1 = \begin{pmatrix}
	  1 \\
	  0 \\
	  0 
	\end{pmatrix}$

По собственным векторам определим канонический базис:

$s_1 = \begin{pmatrix}
	  0 \\
	  \sqrt{\frac{1}{2}+\frac{\sqrt{17}}{34}} \\
	  \sqrt{\frac{1}{2}-\frac{\sqrt{17}}{34}}
	\end{pmatrix}$

$s_2 = \begin{pmatrix}
	  0 \\
	  -\sqrt{\frac{1}{2}-\frac{\sqrt{17}}{34}} \\
	  \sqrt{\frac{1}{2}+\frac{\sqrt{17}}{34}}
	\end{pmatrix}$

$s_3 = \begin{pmatrix}
	  1 \\
	  0 \\
	  0
	\end{pmatrix}$

Составляем матрицу S , записывая по столбцам координаты этих векторов:

\begin{center}
	\begin{equation*}
	S = 
	\begin{pmatrix}
	  0 & 0 & 1\\
	  \sqrt{\frac{1}{2}+\frac{\sqrt{17}}{34}} & -\sqrt{\frac{1}{2}-\frac{\sqrt{17}}{34}} & 0\\
	  \sqrt{\frac{1}{2}-\frac{\sqrt{17}}{34}} & \sqrt{\frac{1}{2}+\frac{\sqrt{17}}{34}} & 0
	\end{pmatrix}
	\end{equation*}
\end{center}

7. Находим координаты $x_0, y_0, z_0$ начала канонической
системы координат. Для эллиптического цилиндра:

\begin{equation*}
 \begin{cases}
   0x_0 + 0y_0 + 0z_0 = 0, 
   \\
   0x_0 - 2y_0 + 2z_0 = -2, 
   \\
   0x_0 + 2y_0 - 3z_0 = -2, 
 \end{cases}
\end{equation*}

Получаем одно из решений $x_0 = 0, y_0 = 5, z_0 = 4$.

Запишем формулы для матрицы $S$ и координатного столбца
$s = (0, 5, 4)^{T}$:

\begin{center}
	\begin{equation*}
	\begin{pmatrix}
	  x \\
	  y \\
	  z
	\end{pmatrix} = s + S
	\begin{pmatrix}
	  x' \\
	  y' \\
	  z'
	\end{pmatrix} = 
	\begin{pmatrix}
	  0 \\
	  5 \\
	  4
	\end{pmatrix} + 
	\begin{pmatrix}
	  0 & 0 & 1\\
	  \sqrt{\frac{1}{2}+\frac{\sqrt{17}}{34}} & -\sqrt{\frac{1}{2}-\frac{\sqrt{17}}{34}} & 0\\
	  \sqrt{\frac{1}{2}-\frac{\sqrt{17}}{34}} & \sqrt{\frac{1}{2}+\frac{\sqrt{17}}{34}} & 0
	\end{pmatrix}
	\begin{pmatrix}
	  x' \\
	  y' \\
	  z'
	\end{pmatrix}
	\end{equation*}
\end{center}

\begin{equation*}
 \begin{cases}
   x = z',
   \\
   y = \sqrt{\frac{1}{2}+\frac{\sqrt{17}}{34}}x' - \sqrt{\frac{1}{2}-\frac{\sqrt{17}}{34}}y' + 5,
   \\
   z = \sqrt{\frac{1}{2}-\frac{\sqrt{17}}{34}}x' + \sqrt{\frac{1}{2}+\frac{\sqrt{17}}{34}}y' + 4
 \end{cases}
\end{equation*}

\begin{sagesilent}
g1 = sqrt(1/2 + sqrt(17)/34)*x - (sqrt(1/2 - sqrt(17)/34))*y + 5
g2 = sqrt(1/2 - sqrt(17)/34)*x + (sqrt(1/2 + sqrt(17)/34))*y + 4
\end{sagesilent}

\begin{center}
  \sageplot[width=.75\textwidth]{implicit_plot3d(f.subs({y: g1, z: g2}), (x, -5, 5), (y, -5, 5), (z, 0, 11))}
\end{center}